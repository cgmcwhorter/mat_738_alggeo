% !TEX root = ../../mat_738_alggeo.tex
\newpage
\section{Projective Varieties}
\subsection{Introduction}

Graded rings first.

\begin{dfn}[Graded Ring]
Let $R$ be a ring. A grading of $R$ is an expression of the additive group $(R,+)$ as an internal direct sum $R= \oplus_{i=1}^\infty R_i$ with the property that if $a \in R_d, b \in R_e$, then $ab \in R_{e+d}$, written $R_e \cdot R_d \subseteq R_{de}$. A graded ring is a ring together with a given grading. An element of $R_d$ is said to be homogeneous of degree $d$. 
\end{dfn}


\begin{ex}
Let $R= k[x_1,\ldots,x_n]$. $R_i= \{ f \in R \colon \text{all monomials of } f  \text{ having deg }i\} \cup \{0\}$, is homogenous of every degree. $\C[x,y]$, $x^2+4xy-17y^2$ is homogeneous of degree 2. $x^3-xy$ is not homogeneous. \xqed
\end{ex}


\begin{rem}
You can replace $\oplus_{i=0}^\infty R_i$ with $\oplus_{i \in M} R_i$, where $M$ is any monoid. We say the case of $\Z$ is $\Z$-grading. 
\end{rem}


Let $R$ be a graded ring. Each $f \in R$ has a unique expression $f= f_0 + \cdots + f_d$, where $f_i$ is homogeneous of degree $i$---simply the definition of a direct sum. 


















Definition and prop (homogeneous graded ideal)


\begin{prop}
Let $R$ be a graded ring and $I \subseteq R$ be an ideal. The following are equivalent:
\begin{enumerate}[(i)]
\item $I$ can be generated by homogeneous elements
\item $I= \oplus_{i=0}^\infty (I \cap R_i)$ as a group
\item given $f \in R$, write $f= f_0 + \cdots + f_d$, where $f_i$ is homogeneous of degree $i$. Then $f \in I$ if and only if all $f_i \in I$. 
\end{enumerate}
\end{prop}

\pf (b) if and only if (c) definition of direct sum. In (c), note that all $f_i \in I$ certainly $f \in I$. The only things to check is $f \in I$, then $f_i \in I$. (c) to (a): write every $f \in I$ as $f= f_0 + \cdots + f_d$, where $f_i$ is homogeneous of degree $i$. Certainly, $I$ is generated by all the $f_i$ that appear as $f$ varies over all $f \in I$. (a) to (c): Say $I$ is generated by $\{f_\alpha\}$, $\alpha \in A$, where $f_\alpha$ is homogenous of degree $d_\alpha$. Choose $F \in I$, then $F= \sum_{i=1}^n a_i f_{\alpha_i}$. Write $F= F_0 + \cdots + F_d$, where $F_j$ is homogeneous of degree $j$. What is $F_j$ in terms of the expression?
	\[
	F_j= \sum_{i=1}^n ((\text{deg }j - \text{deg}(f_{\alpha_i})) \text{ piece of }a_i) f_{\alpha_i \in I}. 
	\] \qed \\


\begin{prop}
Let $R$ be a graded ring.
\begin{enumerate}[(i)]
\item if $I, J$ are homogeneous ideals, then so are $I+J$, $IJ$, $I \cap J$, and $\sqrt{J}$
\item let $I$ be a homogeneous ideal. Then $I$ is prime if and only if for any two homogeneous elements $f,g \in R$, if $fg \in I$ then either $f \in I$ or $g \in I$. 
\item if $f,g \in R$, and integral domain, with $f,g$ nonzero, then $fg$ is homogeneous if and only if $f,g$ are homogeneous. 
\end{enumerate}
\end{prop}


\pf 

$f= f_0 + \cdots + f_d$, $g= g_0 + \cdots+ g_e$
$fg= (f_dg_e) + (f_{d-1}g_e+ f_dg_{e-1}) + \cdots$
If all $f_i \in I$, then $F \in I$ and we are done.
If all $g_i \in I$, then $g \in I$ and we are done.
So assume that $f_i \notin I$ and some $g_j \notin I$.
Let $s$ be the largest number such that $f_s \notin I$ and let $t$ be the largest number such that $g_t \notin I$. We examine the degree $s+t$ piece of $fg$ as $fg \in I$ and $I$ is homogeneous, it is in $I$. It will have the form $f_sg_t+$ terms of the form $f_ig_j$ with either $i>s$ or $j>t$. All of these extra terms are in $I$, the whole sum is in the ideal. Thus, $f_sg_t \in I$ then either $f_s \in I$ or $g_t \in I$, a contradiction. \qed \\


For a homogeneous ideal $I$ of $R$, we often denote $I \cap R_i$ by $I_i$. It then follows that 
	\[
	R/I \cong \dfrac{\oplus R_i}{\oplus I_i} \cong \oplus_{i=0}^\infty R_i/I_i.
	\]
Thus, $R/I$ is naturally a graded ring with $(R/I)_i = R_i/I_i$. Now suppose $R$ is a noetherian graded ring and $I \subset R$ is a homogeneous ideal. We know $I$ can be generated by homogeneous ideal because it is homogeneous. Because $R$ is noetherian, $I$ can be generated by finitely many elements. Can it be generated by finitely many homogeneous elements? Indeed, this is the case. If you set up the proof of the ACC then every ideal is f.g., you can prove ACC then every set of generators for an ideal has a finite subset that generates. 



Now for projective space, you may have heard `parallel lines meet at infinity'. 

% Give image of parallel railroad tracks meeting at infinity. 

The basic idea is to `compactify' $\A^n$ by adding `points at infinity'. This is similar to the way $\R^1$ is compactified to $S^1$ or $\R^2$ to $S^2$. This is not at all apparent from the initial definition of porjective space, but will eventually become clear. 


Consider $\A^{n+1}$ with coordinates $x_0,x_1,\ldots,x_n$. On $\A^{n+1} \setminus \{(0,\ldots,0)\}$, we define the following equivalence relation: $(a_0,\ldots,a_n) \sim (b_0,\ldots,b_n)$ if and only if there exists $\lambda \in k^\times$ such that $(a_0,\ldots,a_n)= (\lambda b_0,\ldots,\lambda b_n)$. The equivalence class of $(a_0,\ldots,a_n)$ is the one-dimensional subspace of $\A^{n_1}$ spanned by $(a_0,\ldots,a_n)$---minus the origin, which is the line through the origin and $(a_0,\ldots,a_n)$ minus the origin. 


\begin{dfn}
Projective $n$-space over $k$, denoted $\bP_k^n$ or $\bP^n(k)$ or $\bP= \A_k^{n+1} \setminus \{(0,\ldots,0)\}/\sim$. A point $P \in \bP$ is an equivalence class of some point $(a_0,\ldots,a_n) \in \A^{n+1} \setminus \{(0,\ldots,0)\}$ and is denoted $[a_0,\ldots,a_n]$. The $a_i$ are called the homogeneous coordinates of the point $P$. They are well defined up to a nonzero constant multiple. $\bP$ is also the set of lines through the origin in $k^{n+1}$, the set of one-dimensional subspaces of $k^{n+1}$. 
\end{dfn}



Denote the polynomial ring $k[x_0,\ldots,x_n]$ by $S$. For $f \in S$ and $O \in \bP^n$, $f(P)$ is not well defined because the coordinates of $P$ are not well defined. But suppose $f \in S$ is homogeneous: $f(\lambda a_0,\ldots, \lambda a_n) = \lambda^{\deg f} f(a_0,\ldots,a_n)$. $f(P)$ still not well defined since $\lambda \in k^\times$, whether $f(P)=0$ is well defined. So we define for a homogeneous $f \in S$, we define the zeros of $f$, denoted $Z(f)$ by $Z(f)= \{ P \in \bP^n \colon f(P)= 0\}.$ For a set of homogeneous elements $T \subseteq S$, we define the zeros of $T$, denoted $Z(T)$ by $Z(T)= \cap_{f \in I} Z(f)= \{P \in \bP^n \colon f(P)=0 \text{ for all }f \in T\}$. A set $Y \subseteq \bP^n$ is called algebraic if and only if $Y= Z(T)$ for some subset $T \subseteq S$ of homogeneous elements. If $I \subseteq S$ is a homogeneous ideal, we define $Z(I)= \{ P \in \bP^n \colon f(P)=0 \text{ for all homogeneous } f \in I\}= \cap_{f \in I, f homog} Z(f)$.


\begin{prop}
Let $T \subseteq S$ be a set of homogeneous elements, and let $I \subseteq S$ be the homogenous ideal generated by $T$. Then $Z(T)= Z(I)$. Thus, every algebraic set in $\bP^n$ is of the form $Z(I)$ for a homogeneous ideal and $Z(\{(f_1,\ldots,f_r)\})$ for finitely many homogeneous $f_i$, $Z(f_1,\ldots,f_r)$. 
\end{prop}


The algebraic subsets of $\bP^n$ satisfy the properties needed to be the closed subsets of a topology on $\bP^n$, it is the Zariski topology. 


\begin{dfn}
A projective algebraic variety (or simply projective variety) is an irreducible algebraic set in $\bP^n$, with the induced topology. An open subset of a projective variety is a quasi-projective variety. The dimension of a variety or quasi-projective variety is its dimension as a topological space. If $Y$ is any subset of $\bP^n$, we define the homogeneous ideal of $Y$ in $S$, denoted $I(Y)$, generated by $\{ f \in S \colon f \text{ homogeneous and } f(P)=0 \text{ for all }P \in Y\}$. If $Y$ is an algebraic set, we define the set of homogeneous coordinate ring of $Y$ to be $S(Y):= S/I(Y)$.
\end{dfn}


Note that $I(Y)$ is a homogeneous ideal so $S(Y)$ is a graded ring. These are similar to old algebraic sets with one funny difference. $S$ is clearly a homogeneous ideal of $S$. $Z(S)= \emptyset$, $S= k[x_0,\ldots,x_n]$ $(x_0,\ldots,x_n)$ is also clearly a homogeneous ideal of $S$. It is often called $S_+$ because it contains all homogeneous elements of positive degree $Z(S_+)= \emptyset$. This is the only thing which goes awry. $S_+$ is sometimes called the irrelevant maximal ideal since it removed from the correspondence we saw in the affine case. 


We now work on showing $\bP^n$ is $\A^{n+1}$ with points added at $\infty$. In $\bP^n$, consider closed sets of the form $Z(f)$ with $f$ a nonconstant polynomial These are called hypersurfaces. When $f$ is linear, it is called a hyperplane. Of particular interest is when $f= x_i$ for some $i$. Set $H_i= Z(x_i)= \{[a_0,\ldots,a_n] \;|\; a_i=0 \}$ and $U_i= \bP^n \setminus H_i$, $i=0,\ldots,n$, $U_i$ open. The $U_i$ are an open cover of $\bP^n$
	\[
	(\cup_i^n U_i)^C= \cap_0^n U_i^C= \cap_0^n H_i= \{[a_0,\ldots,a_N] \colon a_i= 0 \text{ all }i\}= \emptyset. 
	\]










































%%%%

$P= [a_0,\ldots,a_n] \in U_i$ if and only if $P= [a_0/a_i,\ldots,a_{i-1}/a_i,1,a_{i+1}/a_i,\ldots,a_n/a_i]$. Each point in $U_i$ has a unique set of homogeneous coordinates such that the $i$th coordinate is 1. Define a map $\phi_i: \A^n \to U_u$  by $\phi_i(a_1,\ldots,a_n)= [a_1,a_2,\ldots,1,\ldots,a_n]$. Moreover, this is clearly a bijection. 



\begin{prop}
The map $\phi_i$ is a homeomorphism of $U_i$ with its induced topology to $\A^n$ with its Zariski topology. 
\end{prop}

The proof is based on homogenization and dehomogenization of polynomials. To make notation simpler, we assume $i=0$. Let $S= k[x_0,\ldots,x_n]$, $A= k[x_1,\ldots,x_n]$, and $S^h$ denote the homogeneous elements of $S$. There is a ring homomorphism $\alpha: S \to A$, defined as evaluation at $x_0= 1$: $\alpha(f(x_0,\ldots,x_n))= f(1,x_1,\ldots,x_n)$. Since $\alpha$ is a ring homomorphism
	\[
	\begin{split}
	\alpha(f+g)&= \alpha(f)+\alpha(g) \\
	\alpha(fg)&= \alpha(f)\alpha(g),
	\end{split}
	\]
called dehomogenization because even if $f$ is homogeneous, $\alpha(f)$ may not be: $\alpha(x_0^2+x_0x_1)= 1+x_1$. 


Homogenization: $\beta: A \to S^h$: $\beta(f(x_1,\ldots,x_n))= x_0^{\deg f} f(x_1/x_0,\ldots,x_n/x_0)$, $\beta(f)$ is homogeneous of degree $\deg f$. Look at a monomial of $f$: $x_1^{i_1} x_2^{i_2} \cdots x_n^{i_n}$, $i_1+\cdots+i_n \leq \deg f$.
	\[
	x_0^{\deg f} (x_1/x_0)^{i_1} \cdots (x_n/x_0)^{i_n}= x_0^{\deg f - (i_1+\cdots+i_n)} x_1^{i_1} \cdots x_n^{i_n}
	\]
has degree $\deg f$.


$\beta(fg)= \beta(f) \beta(g)$.
$\beta(fg)= x_0^{\deg f + \deg g} f(x_1/x_0,\ldots,x_n/x_0) g(x_1/x_0,\ldots,x_n/x_0)= x_0^{\deg f} f(x_1/x_0,\ldots,x_n/x_0) x_0^{\deg g} g(x_1/x_0,\ldots,x_n/x_0)= \beta(f) \beta(g)$. $\beta(f+g)$ might not equal $\beta(f) + \beta(g)$. 

If $\deg f \neq \deg g$, $\beta(f) + \beta(g)$ will not even be homogeneous. 

If $\deg f= \deg g= \deg(f+g)$, then $\beta(f+g)= \beta(f) + \beta(g)$. 
	\[
	\alpha(\beta(f))= \alpha(x_0^{\deg f} f(x_1/x_0,\ldots,x_n/x_0)= 1^{\deg f} f(x_1/x_0,\ldots,x_n/x_0)= f
	\]
$\beta(\alpha(f))= \beta(\alpha(x_0^l f))$. Assume you have a homogeneous $F$ that is not divisible by $x_0$.
	\[
	F= \sum a_{i_0 \cdots i_n} x_0^{i_0} \cdots x_n^{i_n}
	\]
where $i_0 + \cdots + i_n= d$. 


$\alpha(F)= \sum a_{i_0 \cdots i_n} x_1^{i_0} \cdots x_n^{i_n}$ and in at least one case $i_1+ \cdots + i_n = d$. 

$\beta(\alpha(F))= x_0^d a_{i_0 \cdots i_n} ((x_1/x_0)^{i_0} \cdots (x_n/x_0)^{i_n}= F$. Consider a point $(a_1,\ldots,a_n) \in \A^n$. $\phi_0(a_1,\ldots,a_n)= [1,a_1,\ldots,a_n] \in U_0 \subseteq \P^n$. 

$f \in k[x_1,\ldots,x_n]$
$F \in k[x_0,\ldots,x_n]$ $F$ homogeneous
$f(a_1,\ldots,a_n)= 1^{\deg f} f(a_1/1,\ldots,a_n/1)= \beta(f)(1,a_1,\ldots,a_n)$.
$F(1,a_1,\ldots,a_n)= \alpha(F)(a_1,\ldots,a_n)$.
$f$ vanishes at $P \in \A^n$ if and only if $\beta(f)$ vanishes on $\phi_0(P) \in U_0$
$F$ vanishes at $\phi_0(P) \in U_0$ if and only if $\alpha(F)$ vanishes at $P \in \A^n$.

It is now easy to see that $\phi_0: \A^n \to U_0 \subseteq \P^n$ is a homeomorphism. 

Suppose $Y \subseteq \A^n$ is closed. $Y= Z(T)$, $T \subseteq k[x_1,\ldots,x_n]$. Set $\beta(T)= \{ \beta(f) \;|\; f \in T \}$, $\phi_0(Y)= Z(\beta(T)) \cap U_0$. 
$W \subseteq U_0$ closed in $U_0$
$W= U_0 \cap \overline{W}$, $\overline{W}$ closed in $\P^n$
$\overline{W}= Z(T)$, $T \subseteq S^h$
$\alpha(T)= \{ \alpha(F) \;|\; F \in T\}$
$\phi_0^{-1}(W)= Z(\alpha(T))$


\begin{cor}
If $Y$ is a projective (respectively quasi-projective) variety, then $Y$ is covered by the open sets $U_i \cap Y$, $i= 0,\ldots,n$, which are homeomorphic to affine (respectively quasi-affine) varieties with the map $\phi_i$. 
\end{cor}


$\A^n \cong U_i \subseteq \P^n$, $U_i= \P^n \setminus H_i$, $H_i= Z(x_i)= \{ [a_0,\ldots,0,\ldots,a_n \;|\; \text{not all }a_i=0 \} \cong \P^{n-1}$, $\P^n= \A^n \cup \P^{n-1}$. Parallel lines meet at infinity. 

%%% Ex


$\A^2$ coordinates $x,y$
$Ax+By+C=0$ at least one of $A,B \neq 0$.
$AB+Cy+D=0$, $C \neq D$.

Homogenize 
$Z(A(X/Z)+B(Y/Z)+C=0)$
$Ax+By+CZ=0-Ax+BY+DZ=0$ so $(C-D)Z=0$ so $Z=0$. 
If $Z=0$, still have $Ax+By= 0$. 

$x= \lambda B$, $y= -\lambda A$, $[\lambda B, -\lambda A,0]$ for $\lambda \neq 0$. One point in $\P^2$. $[B,-A,0]$ it is in $H_2$. $x_0 \iff x$, $x_1 \iff y$, $x_2 \iff z$.

Slope of $Ax+By+C=0$ is the ratio $-A/B$. So each slope determines what point at infinity the line goes to. 


%%% Ex

Looking at a curve in different patches and seeing how they fit together. 
Hyperbola: $xy-1=0$
homogeneize that $xy-z^2=0$
$z=1$, $xy-1=0$
$y=1$ $x-z^2=0$
$x=1$ $y-z^2=0$

% Give plots of hyperbola, plot and darken points (1,1), (-1,-1) and (1/2,2), (-1/2,-2)

$B(1/2,2), [1/2,2,1] $
$C[-1,-1,1], (-1,-1)$
$D[-1/2,-2,1], (-1/2,-2)$
$A[1,1,1],(1,1)$

$y- z^2$ (plot)
$E[1,0,0] (0,0)$
$A(1,1)[1,1,1]$
$B(4,2)[1/2,2,1]=[1,4,2]$
$C[-1,-1,1]=[1,1,-1]$
$D[-1/2,-2,1]=[1,4,-2]$

$x-z^2=0$ (plot)
$F(0,0)$
$A(1,1)[1,1,1]$
$B(1/4,1/2), [1/2,2,1]=[1/4,1,1/2]$
$C(1,-1)[-1,-1,1]= [1,1,-1]$
$D(1/4,-1/2)[-1/2,-2,1]=[1/4,1,-1/2]$



%%% Ex 2.12
$\rho_d(P)= [M_0(P), M_1(P), \ldots, M_N(P)]$
First note that the map is well defined.
$\rho_d([\lambda a_0,\ldots, \lambda a_n])= \lambda^d \rho_d([a_0,\ldots,a_n])$ because all of the $M_i$ are homogeneous of degree $d$. 

$f: \P^2 \to \P^2$
$f[(x_0,x_1,x_2])= [x_0x_1x_0x_2,x_0^2]$
$[0,0,1] \to [0,0,0]$, bad 

But not a problem above since one of $a_i$ is nonzero. The exercise asks you to check various things. $\rho_d$ is one-to-one, $\rho_d(\P^n)$ is a closed subset of $\P^N$, if we give $\rho_d(\P^n)$ its induced topology as a subset of $\P^N$ then $\rho_d: \P^n \to \rho_d(\P^n)$ is a homeomorphism. It seems reasonable to call it an embedding.

\begin{ex}
$n=1$, $d=2$
$[a_0,a_1] \mapsto [a_0^2,a_0a_1,a_1^2]$, labeled $X,Y,Z$. The image is contained in $Z(XZ-Y^2)$. In fact,.... \xqed
\end{ex}

\begin{ex}[Segre Embedding]
Let $\psi: \P^r \times \P^s \to \P^n$ be the map defined by sending the ordered pair $[a_0,\ldots,a_r] \times [b_0,\ldots,b_s] \mapsto [\ldots,a_ib_j,\ldots]$ all $a_ib_j$. $N= rs+r+s= (r+1)(s+1)-1$. This is well defined. $[\lambda a_0,\ldots,\lambda a_r] \times [\mu b_0,\ldots,\mu b_s]= \lambda \mu [\cdots a_ib_j \cdots]$. \xqed
\end{ex}


\begin{ex}
$\psi: \P^1 \times \P^1 \to \P^3$
$[a_0,a_1] \times [b_0,b_1] \mapsto [a_0b_0,a_0b_1,a_1b_0,a_1b_1]$ labeled $WXYZ$.
image lies on $WZ-XY=0$ in fact $=$. Book asks you to show $\psi$ is one-to-one and image is closed.  \xqed
\end{ex}


The book does not ask you to show that $\psi$ is a homeomorphism onto its image when you give the image the induced topology. Why? We have not defined the Zariski topology on $\P^r \times \P^s$. 

It is an exercise to show that if we identify $\A^2$ with $\A^1 \times \A^1$ in the natural way that the Zariski topology on $\A^2$ is not the product topology of the Zariski topologies on the two $\A^1$'s. Hint: In $\A^2$ with coordinates $x,y$, $Z(x-y)$ is closed in $\A^2$ with the Zariski topology, but not the product topology. 








































There are three ways to define the Zariski topology on $\P^r \times \P^s$. 

\begin{enumerate}[1.]

\item 

$\P^r$ is covered by $U_i$'s with the $U_i$ homeomorphic to $\A^r$. $\P^s$ is covered by $U_j$'s with each $U_j$ homeomorphic to $\A^s$. $\P^r \times \P^s$ will be covered by $U_i \times U_j$'s $\P^s$ is covered by $U_j$'s with each $U_j$ homeomorphic to $\A^s$. $\P^r \times \P^s$ will be covered by $U_i \times U_j$'s. Give $U_i \times U_j$ the Zariski topology of $\A^{r+s}$, $U_i \times U_k \cap U_k \times U_l$. The topology induced from $U_i \times U_j$ is the same as the topology induced from $U_k \times U_l$. So you can glue to get a topology on $\P^r \times \P^s$. $X \subseteq \P^r \times \P^s$ closed if and only if $X \cap U_i \times U_j$ closed all $i,j$. 

\item on $\P^r$ take homogeneous coordinates $x_0,\ldots,x_r$ and on $\P^s$ take homogeneous coordinates $y_0,\ldots,y_s$. A polynomial $f \in k[x_0,\ldots,x_r,y_0,\ldots,y_s]$ is said to be bihomogeneous of bidegree $d,e$ if and only if every monomial of $f$ has degree $d$ in the $x_i$'s and degree $e$ in the $y_j$'s. In this case, $f(\lambda a_0,\ldots, \lambda a_r,\ldots,\mu b_0,\ldots,\mu b_r)= \lambda^d \mu^e f(a_0,\ldots,a_r,b_0,\ldots,b_r)$ whether $f$ vanishes on $(P,Q) \in \P^r \times \P^s$ is well defined. Define $Y \subseteq \P^r \times \P^s$ is closed if and only if $Y= Z(T)$ for some subset $T \subseteq k[x_0,\ldots,y_s]$ of bihomogeneous polynomials.


\item Give $\P^r \times \P^s$ the unique topology it must have for the Segre map $\psi: \P^r \times \P^s \to \psi(\P^r \times \P^s) \subseteq \P^N$ to be a homeomorphism when $\psi(\P^r \times \P^s)$ is given the induced topology as a subset of $\P^N$.
\end{enumerate}


All three give the same topology. 





% Section 3: morphisms


Note: any affine variety is also a quasi-affine variety. Same thing holds for projective and quasi-projective.


\begin{dfn}[Regular]
Let $Y$ be a quasi-affine variety in $\A^n$. A function $f: Y \to k$ is regular at a point $P \in Y$ if and only if there is an open neighborhood $U$ with $P \in U \subseteq Y$ and polynomials $g, h \in A= k[x_1,\ldots,x_n]$ such that $h$ is nowhere zero on $U$ and $f= g/h$ on $U$. We say that $f$ is regular on $Y$ if and only if it is regular at every point of $Y$.
\end{dfn}

``A function is regular if and only if it is locally a rational function.''

Note: It is important to remember this is not iff continuous does not imply regular. $f: \A^1 \to \A^1$ every bijection is continuous. 








\begin{lem}
A regular function is continuous when $k$ is identified with $\A^1_k$ with its Zariski topology. 
\end{lem}

\pf It is enough to show that $f^{-1}(C)$ is closed for any closed set $C$. First, note that $f^{-1}(\emptyset)= \emptyset$. Now the (nonempty) closed sets of $\A_k^1$ are a finite collection of points. Because of this and the fact that a finite union of closed sets is closed, it suffices to show that $f^{-1}(a)$ is a closed set, i.e. a finite collection of points, where $a \in \A_k^1$. 

This can be checked locally: a subset $Z$ of a topological space $Y$ is closed if there exists an open cover of $Y$ such that $Z \cap U$ is closed in $U$ for every $U$ in cover. Let $U$ be an open set on which $f$ can be represented $f= g/h$, where $g, h$ polynomials on $U$ and $h \neq 0$. The collection of all such $U$'s form an open cover of $\A_k^1$. Observe $f^{-1}(a) \cap U= \{ P \in U \colon g(P)/h(P)= a\}$. But $g(P) / h(P)= a$ if and only if $(g - ah)(p)= 0$. This means $f^{-1}(a) \cap U= Z(g - ah) \cap U$, which is a closed set. Therefore, $f^{-1}(a)$ is closed in $Y$, as desired. \qed \\








Functions on projective spaces are more tricky. $P \in \P^n$, $f \in k[x_0,\ldots,x_n]$. $f(P)$ is not well defined even if $f$ if homogeneous. $f,g \in k[x_0,\ldots,x_n]$ both continuous functions of the same degree $d$. also assume $g(P) \neq 0$. $P= [a_0,\ldots,a_n]$

	\[
	\dfrac{f(\lambda a_0,\ldots,\lambda a_n)}{g(\lambda a_0,\ldots,\lambda a_n)}= \dfrac{\lambda^d f(a_0,\ldots,a_n)}{\lambda^d g(a_0,\ldots,a_n)}= \dfrac{f(a_0,\ldots,a_n)}{g(a_0,\ldots,a_n)}.
	\]
Therefore, $f/g(P)$ is well defined. 


Def: Let $Y$ be a quasi projective variety in $\P^n$. A function $f: Y \to k$ is regular at a point $P \in Y$ iff there is an open neighborhood $Y$ with $P \in U \subseteq Y$ and homogeneous polynomials $g,h \in S= k[x_0,\ldots,x_n[$ of the same degree such that $h$ is nowhere zero on  $Y$ and $f=g/h$ on $Y$. We say $f$ is regular iff it is regular at every point. 


\begin{prop}
Let $Y \subseteq \P^n$ be a quasi-projective variety. $f: Y \to k$ a function. $P \in Y$ and assume $P \in U_i= \P^n \setminus Z(x_i)$. Then when we think of $U_i$ as $\A^n$, $Y \cap U_i$ is a quasi-affine variety and by restriction we get a function $f\big|_{Y \cap U_i}: Y \cap U_i \to k$. Then $f$ is regular at $P$ under the projective definition iff $f\big|_{Y \cap U_i}$ is regular at $P$ under the affine definition. 
\end{prop}

\pf Assume $f$ is regular at $P$ under the projective definition. Thus, we have an open set $U$ with $P \in U \subseteq Y$ and homogeneous polynomials $g,h \in S= k[x_0,\ldots,x_n]$ of the same degree such that $h$ is nowhere zero on $U$ and $f= g/h$ on $U$. On $U_0 \cap U$, which is open in $U_0 \cap Y$
	\[
	\dfrac{g(a_0,\ldots,a_n)}{h(a_0,\ldots,a_n)} = \dfrac{g(1,a_1/a_0,\ldots,a_n/a_0)}{h(1,a_1/a_0,\ldots,a_n/a_0)}.
	\]
But $g(1,x_1,\ldots,x_n) h(1,x_1,\ldots,x_n) \in k[x_1,\ldots,x_n]$, $h \neq 0$ on $U_0 \cap Y$ so $f\big|_{U_0 \cap Y}$ is regular at $P$ under the affine defintion. 

Assume $f\big|_{U_0 \cap Y}$ is regular at $P$ under the affine definition. There is an open neighborhood $U$ with $P \in U \subseteq U_0 \cap Y$ and polynomials $g,h \in k[x_1,\ldots,x_n]$ such that $h$ is nowhere zero on $U$ and $f= g/h$ on $U$. Note that $U$ is open in $Y$ not just $U_0 \cap Y$. Let $G$ and $H$ be the homogenizations of $g$ and $h$. If they do not have the same degree multiply the one of lower degree by an appropriate power of $x_0$ to make them both have the same degree. Now call them $\overline{G}, \overline{H}$. For any point $P= [1,a_1,\ldots,a_n] \in U$, 	\[
	f(P)= \dfrac{g(P)}{h(P)}= \dfrac{\overline{G}(1,a_1,\ldots,a_n)}{\overline{H}(1,a_1,\ldots,a_n)}= \dfrac{\overline{G}(\lambda, \lambda a_1,\ldots, \lambda a_n)}{\overline{H}(\lambda, \lambda a_1,\ldots, \lambda a_n)}
	\]
Fact: $X$ a variety $f,g$ regular functions on $X$. Suppose $f=g$ on some nonempty open $U \subseteq X$. Then $f=g$ on all $X$. Proof $Z(f-g)$ is closed in $X$ and since in an irreducible space nonempty opens are dense it is also dense. $Z(f-g)= X$. \qed \\


\begin{dfn}
Let $k$ be a fixed algebraically closed field. A variety over $k$ is an affine, quasi-affine, projective, or quasi-projective variety. If $X$ and $Y$ are two varieties, a morphism $\phi: X \to Y$ is a continuous map such that for every open $V \subseteq Y$ and for every regular function $f: V \to k$, the function $f \circ \varphi: \varphi^{-1}(V) \to k$ is regular. A morphism $\varphi: X  \to Y$ is a isomorphism if and only if there exists a morphism $\psi: Y \to X$ such that $\varphi \circ \psi= 1_Y$ and $\psi \circ \varphi= 1_X$. 
\end{dfn}


Note that an isomorphism is bijective and bicontinuous (hence a homeomorphism). The converse does not hold. There exists bijective bicontinuous function that are morphisms in one direction but not in the other. 


\begin{ex}
$X= \A_t^1$, $Y= Z(Y^2-X^3) \subseteq \A_{X,Y}^2$
$\varphi: X \to Y$, $\varphi(t)= (t^2,t^3)$
% Give picture
One can check that $\varphi$ is a morphism, one-to-one, onto, $\varphi^{-1}$ exists. Check that it is continuous. $\varphi^{-1}$ is not a morphism. There exists a regular function $f: X \to k$ such that $f \circ \varphi^{-1}: ? \to ?$ is not regular. $f$ is locally a rational function but $f \circ \varphi^{-1}$ is not. Problem at the cusp.
\end{ex}

Alternate definition of a morphism: 

\begin{prop}
Let $X \subseteq \P^n$ and $Y \subseteq \P^m$ be quasi-projective varieties and $f: X \to Y$ a function. Then $f$ is a morphism if and only if the following condition is satisfied: given any $P \in X$, there exists a neighborhood $U$ of $P$, $P \in U \subseteq X$, such that $f(U)$ is contained in one of the affine opens $U_i \subseteq \P^m$. 
\end{prop}

Denote this set by $U_0$. Now think of $f: U \to \A^m$. We require further that there exist $m$ regular functions $g_i: U \to k$ such that for all $Q \in U$
	\[
	f(Q)= (g_1(Q),\ldots, g_m(Q))
	\]
Further since regular functions are locally rational by shrinking $U$, you may assume each $g_i$ is a rational function.


A morphism is a function locally given by tuples of rational functions. We have shown polynomials are continuous in the Zariski topology. Easily follows rational function are continuous where defined. A composition of rational functions is rational.

\begin{ex}
Consider the Veronese $\nu: \P^2 \to \P^2$ given by $\nu([x_0,x_1])= [x_0^2,x_0x_1,x_1^2]$. It is a morphism. $y \mapsto (y,y^2)$. On $U_0 \subseteq \P^1$, $\nu(U_0) \subseteq U_0'$
	\[
	U(1,x_1/x_0)= [1,x_1/x_0, (x_1/x_0)^2]
	\]
$U_1 \subseteq \P^1$, $\nu(U_1) \subseteq U_2'$
$\nu(x_0/x_1,1)= [(x_0/x)^2,x_0/x,1]$
$\nu(y)= [y^2,y]$.
\end{ex}


Let $X \subseteq \P^n$ and $Y \subseteq \P^m$ be quasi-projective varieties and $f: X \to Y$ be a function. Then $f$ is a morphism if and only if the following condition is satisfied: given any $P \in X$, there is an open neighborhood $U$ of $P$. $P \in U \subseteq X$ and $m$ polynomials. $F_0,\ldots,F_m \in k[x_0,\ldots,x_n]$ all homogeneous of the same degree such that for all $Q \in U$, $f(Q)= [F_0(Q), F_1(Q), \ldots, F_m(Q)]$. 


\begin{ex}
We can find the inverse of the Veronese embedding. $\nu: \P^1 \to \P^2$
$\nu([x_0,x_1])= px_0^2,x_0x_1,x_1^2]$ labeled $(Y_0,Y_1Y_2)$
On $U_0 \subseteq \P^2$, 
$\nu^{-1}[Y_0,Y_1,Y_2]= [Y_0,Y_1]$
On $U_2 \cap \nu(\P^1) \subseteq \P^2$
$\nu^{-1}[Y_0,Y_1,Y_2]= [Y_1,Y_2]$
$U_0$ and $U_2$ cover $\nu(\P^1)$

$U_0$: $[x_0,x_1] \mapsto [x_0,x_0x_1,x_1^2] \mapsto [x_0^2,x_0x_1]= [x_0,x_1]$ since $x_0 \neq 0$. 

$U_1$: $[x_0,x_1] \mapsto [x_0,x_0,x_1,x_1^2] \mapsto [x_0x_1,x_1^2]= [x_0,x_1]$ since $x_1 \neq 0$. 


No the other way:

$U_0$: $[Y_0,Y_1,Y_2] \mapsto [Y_0,Y_1] \mapsto [Y_0^2,Y_0Y_1,Y_1^2]$ only working on $\nu(\P^1)$, $Y_0Y_2= Y_1^2$
$[Y_0^2,Y_0Y_1,Y_0Y_2]= [Y_0,Y_1,Y_2]$

$U_1$: $[Y_0,Y_1,Y_2] \to [Y_1,Y_2] \to [Y_1^2,Y_1Y_2,Y_2^2]= [Y_0Y_2,Y_1Y_2,Y_2^2]= [Y_0,Y_1,Y_2]$ since $Y_2 \neq 0$. 

$\nu(\P^1) \cap U_0 \cap U_2$ on $\P^2$
$[Y_0,Y_1,Y_2] \to [Y_0,Y_1]= [Y_0Y_2,Y_1,Y_2]= [Y_1^2,Y_1Y_2] = [Y_1,Y_2]$ using $Y_0Y_2=Y_1^2$. 
\end{ex}








\begin{dfn}[Local Ring of $P$ on $Y$]
Let $Y$ be a variety. We denote by $\O(Y)$ the ring of all regular functions on $Y$. If $P$ is a point of $Y$, we define the local ring of $P$ on $Y$, denoted $\O_{P,Y}$ or simply $\O_P$, to be the ring of germs of regular functions on $Y$ near $P$, i.e. equivalence classes of pairs $\langle U,f \rangle$, where $U$ is an open neighborhood of $P$ in $Y$ and $f$ is a regular function on $U$ with the equivalence relation $\langle U,f \rangle \sim \langle V,g \rangle$ if and only if $f=g$ on $U \cap V$.
\end{dfn}


The ring operations on $\O_{P,Y}$ are given below, though we leave it to the reader that these operations are well defined (make use of the fact that $Y$ is a variety, and hence $Y$ is irreducible). 
	\[
	\begin{split}
	\langle U,f \rangle + \langle V,g \rangle&= \langle U \cap V, f+g \rangle \\
	\langle U,f \rangle \langle V,g \rangle&= \langle U \cap V, fg \rangle 
	\end{split}
	\]
There is more to check. For instance, we have not shown that $\O_{P,Y}$ is a local ring in the `regular' sense of the word, i.e. a ring possessing a unique maximal ideal, or that the relation is indeed an equivalence relation. Again, we leave it to the reader to verify these facts as they are routinely checked. 









































% Missing two lectures here?






%%%%%%%%

Notice that for $\langle U,f \rangle \in \O_{P,Y}$, $f(P)$ is well defined because $P$ is contained in every $U$. The evaluation map $\ev: \O_{P,Y} \to k$ given by $\langle U,f \rangle \mapsto f(P)$ is easily seen to be a ring homomorphism. For surjectivity, observe that $\langle Y, a \rangle \O_{P,Y}$ is the constant function $a$ for $a \in k$. The kernel is a maximal ideal, $\fm_{P,Y}= \{ \langle U,f \rangle \colon f(P)=0 \}$. Moreover, $\O_{P.Y}/ \fm{P,Y} \cong k$. 


\begin{rem}
For any commutative noetherian ring $R$, we say that $R$ is a local ring if and only if it has a unique maximal ideal, $\fm$. We say that $R/\fm$ is the residue field of $R$. 
\end{rem}


\begin{prop}
$\O_{P,Y}$ is a local ring, i.e. $\fm_{P,Y}$ the unique maximal idea. 
\end{prop}

\pf Recall that a ring is local if and only if it has a unique maximal ideal if and only if the only proper maximal ideal consists of all non-units of the ring. Suppose that $\langle U,f \rangle \notin \fm_{P,Y}$. Then $f(P) \neq 0$. By possibly restricting to a smaller open subset, $P \in U_1 \subseteq U$, we may write $f= g/h$, where $g,h \in k[x_1,\ldots,x_n]$, $h \neq 0$ on $U_1$, and $g(P) \neq 0$. Let $U_2= U_1 \setminus Z(g)$. We know that $P \in U_2$ and $h/g$ is regular on $U_2$. But then $\langle U_2, h/g \rangle \in \O_{P,Y}$. But
	\[
	\langle U_2, h/g \rangle \cdot \langle U,f \rangle= \langle U_2, h/g \rangle \langle U_2, g/h \rangle= \langle U_2,1 \rangle,
	\]
so that $\langle U,f \rangle$ is invertible in $\O_{P,Y}$. Therefore, $\langle U,f \rangle$ is not contained in any proper ideal. This shows that $\fm_{P,Y}$ is the unique maximal ideal. \qed \\


Note that the above proof did not show that $\O_{P,Y}$ was indeed noetherian, but we shall see this later. In fact, one does not need commutative or noetherian to define local, merely the unique maximal ideal. However, commutative algebraists typically, when using the term `local ring', intend for this to mean commutative, noetherian, and with unique maximal ideal.


\begin{dfn}
If $Y$ is a variety, we define the function field of $K(Y)$ of $Y$ as follows: an element of $K(Y)$ is an equivalence class of pairs $\langle U,f \rangle$, where $U$ is a nonempty open subset of $Y$, $f$ is a regular function on $U$, and we define $\langle U,f \rangle \sim \langle V,g \rangle$ if and only if $f=g$ on $U \cap V$. We define
	\[
	\begin{split}
	\langle U,f \rangle + \langle V,g \rangle&= \langle U \cap V, f+g \rangle \\
	\langle U, f \rangle \langle V,g \rangle&= \langle U \cap V, fg \rangle.
	\end{split}
	\]
The elements of $K(Y)$ are called rational functions on $Y$.
\end{dfn}


One should check that $K(Y)$ is indeed a field. Note that one needs $Y$ to be irreducible so that any two open sets intersect. Otherwise, the addition and multiplication described above might not make sense. We have natural ring homomorphisms 
	\[
	\begin{tikzcd}
	k \arrow[hook]{r} & \mathcal{O}(Y) \arrow[hook]{r} & \O_{P,Y} \arrow[hook]{r} & K(Y),
	\end{tikzcd}
	\]
with maps $a \mapsto$ constant function $a$, $f \mapsto \langle U, f \rangle$, and $\langle U,f \rangle \langle U,f \rangle$, respectively. It is routine to verify that these are injective ring homomorphisms. Furthermore, $\O(Y)$, $\O_{P,Y}$, and $K(Y)$ are all $k$-algebras. We often think of them as subrings. 


Now let $f: X \to Y$ be a morphism of varieties. Since by definition under a morphism, regular functions pull back to regular functions, it is easy to see we get the following ring homomorphism: $f^*: \O(Y) \to \O(X)$. If $P \in X$, $f(P) \in Y$. Now $f^*: \O_{f(P),Y} \to \O_{P,X}$ is given by $\langle U, g \rangle \mapsto \langle f^{-1}(U), f^*(g) \rangle$ and is a $k$-algebra homomorphism. When $f$ is an isomorphism, these are both isomorphisms. Furthermore, $(f^{-1})^*= (f^*)^{-1}$, and $\O(Y), \O_{P,Y}$ are isomorphic invariants. What about the map $f^*: K(Y) \to K(X)$? This is not always defined: if $f(X)$ is contained in a proper closed subset of $Y$, there will be some open $U \subseteq Y$ such that $f^{-1}(U)= \emptyset$. 



We now give a brief review of localization in commutative rings.

\begin{dfn}
Let $R$ be a ring and $S \subseteq R$. We say $S$ is multiplicatively closed if and only if $1 \in S$ and if $a,b \in S$, then $ab \in S$. 
\end{dfn}


Given a ring $R$ and a multiplicatively closed set $S \subseteq R$, we define the localization of $R$ (with respect to $S$), denoted $S^{-1}R$ or $R_S$ as follows: as a set $S^{-1}R$ is the set of equivalence classes of elements $r/s$, where $r \in R, s \in S$, under the equivalence relation
	\[
	\dfrac{r}{s} \sim \dfrac{a}{t} \Leftrightarrow= \exists w \in S (SUCH THAT) w(rt-sa)=0.
	\]
We make $S^{-1}R$ into a ring by defining
	\[
	\begin{split}
	\dfrac{r}{s} + \dfrac{a}{t}&= \dfrac{rt+as}{st} \\
	\dfrac{r}{s} \cdot \dfrac{a}{t}&= \dfrac{ra}{st}.
	\end{split}
	\]
One of the most common examples of such an $S$ and a ring is when $R$ is an integral domain and $S$ is the prime ideal $S= R \setminus \{0\}$. In this case $S^{-1}R$ is the fraction field of $R$. More generally for any integral domain $R$, if $\fp$ is a prime ideal, then $S:= R \setminus \fp$ is a multiplicatively closed subset (as one easily verifies). In this special case, $S^{-1}R$ is denoted $R_\fp$ instead of $R_S= R_{R \setminus \fp}$, called the localization of $R$ at $\fp$. As another typical example, if $f \in R$ is not nilpotent, then $S= \{1,f,f^2,\ldots\}$ is a multiplicatively closed subset of $R$. In this case, $S^{-1}R$ is denoted $R_f$, called the localization of $R$ at $f$.


\begin{rem}
If $f$ is prime, then $(f)$ is prime. However, $R_f \neq R_{(f)}$. 
\end{rem}


\begin{thm}
Let $Y \subseteq \A^n$ be an affine variety with affine coordinate ring $A(Y)$. Then
\begin{enumerate}
\item $\O(Y) \cong A(Y)$
\item for each point $P \in Y$, let $\fm_P \subseteq A(Y)$ be the ideal of functions vanishing at $P$. Then $P \mapsto \fm_P$ gives a one-to-one correspondence between the points of $Y$ and the maximal ideals of $A(Y)$
\item for each $P$, $\O_{P,Y} \cong A(Y)_{\fm_P}$ and $\dim \O_{P,Y}= \dim Y$
\item $K(Y)$ is isomorphic to the quotient field of $A(Y)$, and hence $K(Y)$ is a finitely generated field extension of $k$ with transcendence degree $\dim Y$. 
\end{enumerate}
\end{thm}

\pf \hfill
\begin{enumerate}[(a)]
\item We have an injective homomorphism $\alpha: A(Y) \to \O(Y)$. Start with the homomorphism $f: k[x_1,\ldots,x_n] \to \O(Y)$. A regular function need to be locally rational (but polynomials are globally rational). The kernel of the map is $I(Y)$. Now
	\[
	\alpha': A(Y)= k[x_1,\ldots,x_n]/I(Y) \ma{} \O(Y)
	\]
is injective. We prove surjectivity later. 


\item The points of $\A^n$ are in bijective correspondence with the maximal ideals of $k[x_1,\ldots,x_n]$. [We know that the maximal ideals are points $P=(a_1,\ldots,a_n)$ and $I(P)= (x_1-a_1,\ldots,x_n-a_n)$, the polynomials vanishing at $P$.] Now $P \in Y$ if and only if $I(P) \supseteq I(Y)$. But $I(P)$ is maximal. The points of $Y$ are then in bijective correspondence with the maximal ideals of $k[x_1,\ldots,x_n]$ containing $I(Y)$ which are in correspondence with the maximal ideals of $A(Y)$, which are $k[x_1,\ldots,x_n]/I(Y)$. 

\item For each point $P$, there is a natural map $A(Y)_{\fm_P} \to \O_{P,Y}$

$a/b \in A(Y)_{\fm_P}$, where $a,b \in A(Y)$ and $b(P) \neq 0$. Replace $a,b$ by polynomials that represent them. Now $Y \setminus Z(b)$ is an open neighborhood of $P$ in $Y$.

$a/b \mapsto \langle Y \setminus Z(b), a/b \rangle \in \O_{P,Y}$ is injective as $\alpha$ is injective.

For surjective, $\langle U,f \rangle \in \O_{P,Y}$
represented by $\langle V, g/h \rangle$, $g,h$ polynomials, $h \neq 0$ on $V$, $g/h \in A(Y)$. 

This gives $\O_{P,Y} \cong A(Y)_{\fm_P}$. 
\end{enumerate}












































%%%%%%%%%%%%%%%%%%%%%%%%%%%%%%%%%%%%%%%

$S: A \to C$
Dense: For any $c \in C$, there is an $a \in A$ such that $Sa$ is isomorphic to $c$.
On objects, surjective up to isomorphism. 


$X \to A(X)$

On objects, surjective up to isomorphism.

$X \to A(X)$, $X$ affine variety. 

affine varieties $\mapsto$ finitely generated integral domains over $k$.

Bijection: $\alpha: \Hom(X,Y) \stackrel{\sim} \Hom(A(Y),\O(X) \cong A(X))$, where used $\O(X) \cong A(X)$ is $X$ is affine. 

full and faithful come from being a bijection.

dense: Remark 1.4.6. Showed that any f.g. integral domain over $k$ was isomorphic to $A(X)$ for some $X$.





Automorphisms of $\P^n$.

Let $M$ be an invertible $(n+1) \times (n+1)$ matrix with entries in $k$. $M$ induces a map $m: \P^n \to \P^n$ via 

$[a_0,\ldots,a_n] \to M \begin{bmatrix} a_0 \\ \vdots \\ a_n \end{bmatrix}$
$[\lambda a_0, \ldots, \lambda a_n] \mapsto \lambda M \begin{bmatrix} a_0 \\ \vdots \\ a_n \end{bmatrix}$

Since $M$ is invertible, the only $[a_0,\ldots,a_n]$ mapping to $[0,\ldots,0]$ is $[0,\ldots,0]$. It's a morphism because if you multiply out $M \begin{bmatrix} a_0 \\ \vdots \\ a_n \end{bmatrix}$, you can see each entry is a homogeneous linear polynomial in the $a_i$.

The inverse morphism is multiplication by $M^{-1}$.

That is an automorphism.

If $D= \begin{bmatrix} d & & & \\ & d & & \\ & & \ddots & \\ \end{bmatrix}$, then $d \neq 0$. $M$ and $DM$ induce the same map: 

$DM \begin{bmatrix} a_0 \\ \vdots \\ a_n \end{bmatrix}= dM \begin{bmatrix} a_0 \\ \vdots \\ a_n \end{bmatrix}$.

$\GL_k(n+1)$ group of all $(n+1) \times (n+1)$ invertible matrices operation matrix multiplication general linear group. 

The set of scalar matrices is a normal subgroup. 

Define $PGL_k(n+1)= \GL_k(n+1)/scalar matrices$. Projective general linear group. 

It can be shown that $\Aut(\P^n_k) \cong PGL_k(n+1)$.

Linear Varieties (Subspaces) in $\P^n$

$\P^n= \A^{n+1} \setminus \{0\} /\sim$, where $\sim$ $(a_0,\ldots,a_n) \sim \lambda (a_0,\ldots,a_n)$, where $\lambda \in k^\times$ which is lines through origin. 

Go back to thinking of $\A^{n+1}$ as $k^{n+1}$ a vector space. 


If $V \subseteq k^{n+1}$ is any subspace of dimension $m+1$. Let $X= \{ P \in \P^n \colon the line to which P corresponds lies in V\}$

$X \cong \P^m$, $V \cong k^{m+1}$

$X$ is called a linear subvariety of $\P^n$, sometimes it is called a linear subspace. This is confusing because $\P^n$ is not a vector space. 

$V \subseteq k^{n+1}$ is the zero locus of $n+1-(m+1)$ homogeneous linear equations.

$X$ will also be the zero set of these equations. 

In fact, they generate its ideal. By change of variables, you assume those equations $x_0= \cdots= x_{(n+1)-(m+1)-1}=0$. Then not hard to see $X$ is called an $m$-plane.

$X \cap U_i \subseteq \A^n$. 

It really is a plane.

Some important exercises in Hartshorne 3.1e., 3.14, 3.15, 21


% Group varieties examples, with topological
% counterexample from screenshot (in alg geo folder)



4. Rational Maps

% Include Lemma 4.1 

% Similar to..... Add remark 3.1.1




\begin{dfn}
Let $X$ and $Y$ be varieties. A rational map 
\end{dfn}















































%%%%%%%%%%%%%%%%%%%%%%%%%



Let's convince ourselves that the statement is true.

We have $\langle U, \varphi_U \rangle$ such that $\varphi_U(U)$ is dense in $Y$. This means that for any nonempty open $V \subset U$, $\varphi_U(U) \cap V \neq \emptyset$. Now suppose we have another pair $\langle W, \varphi_W \rangle$. We need to show that for any open $V \subseteq Y$, $\varphi_U(U) \cap V \neq \emptyset$ and since morphisms are continuous, $\varphi_U^{-1}$ is nonempty open in $U$  and hence open in $X$. This means $\varphi_U^{-1}(V) \cap W \neq \emptyset$ because $X$ is irreducible. [Recall all nonempty open subsets meet.] Choose $P \in \varphi_U^{-1}(V) \cap W$, note $P \in U \cap W$. By definition, $\varphi_U$ and $\varphi_W$ agree on $U \cap W$. 
	\[
	\varphi_W(P)= \varphi_U(P) \in V.
	\]
Show $\varphi_W(W) \cap V \neq \emptyset$.


A rational map might not be defined at some points of $X$. There may be some points of $X$ not in any $U$ such that $\langle U, \varphi_U \rangle$ is in the rational map. 


Example: Projection from a point: $\varphi: \P^n \setminus \{P\} \to \P^{n-1}$. $\langle \P^n \setminus \{P\}, \varphi \rangle$. Sometimes rational maps are denoted by dotted arrows. In a talk, Fulton once said thats because some points (parts of the arrow) do not make it and fall out. This means we cannot always compose rational maps:
	\[
	X \ma{\varphi} Y \ma{\psi} Z
	\]
where $\varphi, \psi$ are rational maps. It could be that $\varphi(X)= U \langle U, \varphi \rangle \varphi_U(U)$ is contained in the set of points where $\psi$ is not defined. This will not happen if $\varphi$ is dominant. 



In fact, you can check that if $\varphi$ and $\psi$ are dominant then so is $\psi \circ \varphi$. So you can make a category the objects are varieties and the morphisms are dominant rational maps. 


\begin{dfn}
A birational map $\varphi: X \to Y$ is a rational map that admits an inverse, namely a rational map $\psi: Y \to X$ such that $\psi \circ \varphi= 1_X$ and $\varphi \circ \psi= 1_X$ as rational maps. (Equal to identity where they are defined). If there is a birational map from $X$ to $Y$, we say that there are birationally equivalent or birational or birationally isomorphic.
\end{dfn}


What is a birational morphism? That is a morphism that has a rational inverse. It is a birational isomorphism. 


Examples: $\A^n$ and $\P^n$ are birationally isomorphic. (They are not isomorphic.) $\varphi: \A^n \to \P^n$ given by $(a_1,\ldots,a_n) \mapsto (1,a_1,\ldots,a_n)$ and $\psi: \P^n \to \A^n$ defined on $U_0$ given by $(1,a_1,\ldots,a_n) \mapsto (a_1,\ldots,a_n)$. 

2) $X= Z(y^2-x^2(x+1)) \subseteq \A^2$ % Give plot, nodal cubic
$X$ is birationally isomorphic to $\A^1$.
$\varphi: \A^1 \to X$
$\varphi(t)= (t^2-1,t(t^2-1))$
$(t(t^2-1))^2 - (t^2-1)^2(t^2-1+1)$
$t^2(t^2-1)^2 -(t^2-1)^2t^2=0$
$\varphi(\A^1) \subseteq X$
$\psi: X \to \A^1$
$\psi(x,y)= y/x$, defined where $x \neq 0$
$\psi \circ \varphi(t)= \psi(t^2-1,t(t^2-1))= \frac{t(t^2-1)}{t^2-1}=1$
away from $t= \pm 1$
$\varphi \circ \psi= \varphi(y/x)= (y^2/x^2-1,y/x(y^2/x^2-1))$ but on $X$ $y^2/x^2= x+1$ as long as $x \neq 0= (x+1-1,y/x(x+1-1))= (x,y)$ away from $x=0$.

Geometric interpretation of the map

Projecting from the origin to the line $x=1$.

% Give picture of nodal cubic, line x=1, then line y= tx t

line $y=tx$ through the origin. It meets $x=1$ at point $(1,t)$. It meets $y^2-x^2(x+1)$ at $t^2x^2-x^2(x+1)=0$
	\[
	x^2(t^2-(x+1))=0
	\]
so $x=0$ twice.

$x= t^2-1$
$y=t(t^2-1)$

Remark: Clearly a morphism is a rational map. Thus, $X$ is isomorphic to $Y$ implies $X$ is birationally isomorphic to $Y$. Those examples show reverse does not hold.

$\O(\A^n)= k[x_1,\ldots,x_n]$
$\O(\P^n)= k$

$\A^n$ not isomorphic to $\P^n$
$A(\A^1)= k[x]$
$A(X)= k[x,y]/(y^2-x^2(x+1))$

$k[x]$ is a PID

In $A(X)$, look at ideal generated by $\overline{x}, \overline{y}$. Anything in $(y^2 - x^2(x+1))$ has all its monomials of degree at least $(\overline{x},\overline{y})= (h(\overline{x},\overline{y}))$ 
$h$would have to have no constant term because $h \in (\overline{x},\overline{y})$. 

$\overline{x}$ and $\overline{y}$ would have to be constant multiples of degree 1 term. Birational is a weaker equivalence relation than isomorphic. Ideally, you would like to classify varieties up to isomorphism. This is \textbf{extremely} difficult. You break it into steps. First, classify up to birational isomorphism, then for each birational equivalence class, try to classify up to isomorphism. 

Confusing terminology.

When we say $X$ is an affine variety, we mean $X$ is a closed irreducible set of some affine space, say $\A^n$. 

When we say a variety $X$ is affine, we mean that $X$ is isomorphic to an affine variety 


Example: $\A^1 \setminus \{0\}$ not an affine variety but it is affine.

$\A_t^1 \setminus \{0\} \cong X= Z(xy-1) \subseteq \A_{x,y}^2$

$\varphi: \A_t^1 \setminus \{0\} \to X$ given by $t \mapsto (t,1/t)$
$\psi: X \to \A^1 \setminus \{0\}$ given by $(x,y) \mapsto x$
$\psi(\varphi(t))= \psi(t,1/t)= t$
$\varphi(\psi(x,y))= \varphi(x)= (x,1/x)$
But on $xy-1=0$, $y=1/x$ so $(x,1/x)=(x,y)$. 


Lemma 4.2: Let $Y$ be a hypersurface in $\A^n$ given by the equation $f(x_1,\ldots,x_n)= 0$. Then $\A^n \setminus Y$ is isomorphic to the hypersurface $H$ in $\A^{n+1}$ given by $x_{n+1}f - 1= 0$. In particular, $\A^n \setminus Y$ is affine and its coordinate ring is $k[x_1,\ldots,x_n]_f$. In previous example, $f=x$ so $x_{n+1}f$ becomes $xy=1$. 

Pf: $\varphi: \A^n \setminus Y \to H$ given by $(x_1,\ldots,x_n)= (x_1,\ldots,x_n,1/f)$

$1/f \cdot f-1=0$
$\psi: H \to \A^n \setminus Y$
$\psi(x_1,\ldots,x_{n+1})= (x_1,\ldots,x_n)$
Check $\varphi \circ \psi= 1$, $\psi \circ \varphi= 1$

$k[x_1,\ldots,x_{n+1}]/(x_{n+1}f-1) = A(H) \cong k[x_1,\ldots,x_n]_f$

as 
$x_{n+1}f-1=0$
$x_{n+1}f=1$
$f= 1/x_{n+1}$. 


$k[x_1,\ldots,x_{n+1}] \to k[x_1,\ldots,x_n]_f$
given by $x_i \mapsto x_i$, $1 \leq i \leq n$
$x_{n+1} \mapsto 1/f$.

Obviously, a surjective ring homomorphism. Show kernel is $(x_{n+1}f-1)$. Kernel contains the ideal is easy. 



























